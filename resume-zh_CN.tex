% !TEX TS-program = xelatex
% !TEX encoding = UTF-8 Unicode
% !Mode:: "TeX:UTF-8"

\documentclass{resume}
\usepackage{zh_CN-Adobefonts_external} % Simplified Chinese Support using external fonts (./fonts/zh_CN-Adobe/)
%\usepackage{zh_CN-Adobefonts_internal} % Simplified Chinese Support using system fonts
\usepackage{linespacing_fix} % disable extra space before next section
\usepackage{cite}
\usepackage[T1]{fontenc}
\usepackage[utf8]{inputenc}

\begin{document}
    \pagenumbering{gobble} % suppress displaying page number

    \name{胡文俊}

    \basicInfo{
    \phone{+86 186-4408-7527} \textperiodcentered\
    \email{geek.hu@qq.com} \textperiodcentered\
    \github[GitHub - benheart]{https://github.com/benheart} \textperiodcentered\
    \faRss\ {\href{https://www.geeeeeeek.com/}{Blog - BenHeart}}}

    \section{\faGraduationCap\  教育背景}
    \datedsubsection{\textbf{哈尔滨工业大学}}{2012.09 - 2016.07}
    \textit{本科 \textperiodcentered\ 计算机学院 \textperiodcentered\ 信息安全}
    \blankline{ }

    \section{\faUsers\ 工作/实习经历}

    \datedsubsection{\textbf{美团点评(上海)} \hfill \textit{Java后台开发工程师}}{2017.07 - 至今}
    \begin{onehalfspacing}
        \datedline{\textbf{到综商户列表收归}}{2020.06 - 2020.09}
        \textbf{背景}:到综休娱、丽人、结婚、学培等多个行业的商户列表页存在重复的业务功能和稳定性建设,且建设质量参差不齐,存在研发效率不高,用户体验不一致的问题。因此,统一收归商户列表页来维护和建设\newline
        \textbf{职责}:负责各Bu商户列表功能需求承接,线上Case排查处理;对现有Bu商户列表功能梳理,归纳总结;调研业界商户列表,共性模块分析抽象;参与统一商户列表平台的设计和搭建。\newline
        \textbf{行动}:
        \begin{itemize}
            \item 架构分层:把现有列表页拆分为聚合场景层和主题场景层。架构分层的意义:1、降低了列表页系统的复杂性;2、便于接口模型标准化,标准化是配置化的基础;3、聚合服务不再关注底层领域模型;4、主题服务避免了不同展示场景中相同展示内容的重复建设问题
            \item 配置化:对平台列表页系统处理流程进行统一抽象,确定易变节点和固化模型,对易变节点进行参数配置化
            \item 高性能:1、下游RPC服务调用全面异步化;2、利用Redis缓存来提高服务性能;3、通过MQ异步更新缓存,减少对主流程影响
            \item 高可用:1、进行周期压测确定单机容量;2、基于单机容量配置合理限流;3、根据服务强、弱依赖划分制定熔断降级方案
        \end{itemize}
        \textbf{收益}:
        \begin{itemize}
            \item 已完成各Bu商户列表功能的交接和需求收归
            \item 已调研淘宝商户列表现有功能和架构设计
            \item 统一商户列表平台的设计和搭建正在进行中
        \end{itemize}
    \end{onehalfspacing}
    \blankline{ }

    \begin{onehalfspacing}
        \datedline{\textbf{页面监控系统}}{2019.05 - 2019.11}
        \textbf{背景}:面向开发和测试人员,支持录入基准页信息并保存基准页面Dom,通过定时任务爬取页面Dom,利用文本比对算法计算两者相似度来判断页面是否有变化,基于预设的告警阈值进行监控告警\newline
        \textbf{职责}:独立负责整个系统的技术调研、方案设计和开发测试,从0到1搭建整套系统\newline
        \textbf{行动}:
        \begin{itemize}
            \item 采用策略模式来支持Jsoup、HtmlUnit、Selenuim三种页面爬取方式
            \item 采用余弦相似度文本比对算法进行Dom文本比对
            \item 通过对Dom分层加权、子节点计数加权进行算法调优
            \item 改造Dom节点Style样式进行页面差异比较
        \end{itemize}
        \textbf{收益}:
        \begin{itemize}
            \item 接入成本极低,仅需配置页面Url和告警接收人,无任何代码侵入
            \item 支持页面模块细粒度黑、白名单配置,人性化的页面差异高亮预览
            \item 已有5条业务线共计60+个页面接入页面监控系统,单页面平均接入时长2-3min
            \item 上线至今,共计发现20+起线上页面故障,及时帮助业务方减少影响范围
        \end{itemize}
    \end{onehalfspacing}
    \blankline{ }

    \begin{onehalfspacing}
        \datedline{\textbf{统一Token加密服务}}{2018.10 - 2019.09}
        \textbf{简介}:手机号、邮箱等铭感信息明文展示及存储风险高,各团队安全意识参差不齐,加密方式各不相同,敏感数据的互通和交换成本较大,无法统一管理维护。为解决这些痛点,我们提供统一Token化服务,对明文加密存储,并生成唯一Token返回\newline
        \textbf{职责}:负责统一Token服务的技术方案选型、开发测试,\newline
        \textbf{行动}:
        \begin{itemize}
            \item 高可用:应用集群(到综、金融、出行)部署隔离;Tair缓存业务隔离,分BG调用;关键链路熔断降级;接入公司Rhnio服务限流、Hulk弹性伸缩
            \item 高性能:数据库分128张表;采用Redis、Tair双缓存;新号段数据预热;LeafId内存队列;Luhn算法快速失败
            \item 安全性:采用AES256-GCM加密算法,秘钥定期轮换;Token与明文不能通过解密或计算获得,防破解、遍历
            \item 易用性:Token长度控制在15位以内,业务方无需做DDL改造;提供Token平台方便业务方自主接入,审核、管理流程化
        \end{itemize}
        \textbf{收益}:
        \begin{itemize}
            \item 全公司5大事业群61个开发团队共计300+应用接入统一Token加密服务
            \item 日均调用量10亿,峰值QPS 2W+,共收录近9亿Token数据
        \end{itemize}
    \end{onehalfspacing}
    \blankline{ }

    \begin{onehalfspacing}
        \datedline{\textbf{常规业务开发}}{2017.7 - 2020.09}
        \textbf{概述}:负责学培商家端、用户端品牌、在线教育、留学等常规业务开发\newline
        \textbf{内容}:
        \begin{itemize}
            \item 品牌重构:废弃ftl模板,采用前后端动静分离重构品牌模块;新增类目属性为品牌垂直化做铺垫;品牌接入诚信、资质审核,先发后审
            \item 在线教育:学培频道页新增在线教育入口;商户品牌、课程信息整合展示;新增品牌、课程索引提升查询效率
            \item 留学专区:全新留学专区频道页开发;新增留学院校、留学国家、留学类目等数据模型;留学院校运营后台功能开发
            \item 频道页改版:金刚位、类目导航、中通位对接海马平台;支持模块可配置,扩展性和复用性;保证高可用,配置监控告警
        \end{itemize}
        \textbf{收益}:
        \begin{itemize}
            \item 品牌重构后,线上品牌月均Case降低70\%
            \item 在线教育上线至今,共计接入37家全国知名在线教育品牌,极大提升业务收入
            \item 留学专区上线至今,共计录入456所全球热门留学院校,新增3种留学商户通
            \item 频道页改版后,首页打开速度提升30\%,用户意向转化率提升3\%
        \end{itemize}
    \end{onehalfspacing}
    \blankline{ }

    \begin{onehalfspacing}
        \datedline{\textbf{团队建设}}{2017.08 - 2020.09}
        \begin{itemize}
            \item 4次技术分享:《网络爬虫》、《分库分表》、《搜索之路》、《统一电话技术总结》
            \item 规范制定:日志规范、压测规范、周报模板、上线规范
            \item 人才培养:帮助两位新人排查线上Case,解答疑惑,分享沟通技巧、工作心得
        \end{itemize}
    \end{onehalfspacing}
    \blankline{ }

    \datedsubsection{\textbf{上海叶道科技有限公司} \hfill \textit{Java开发工程师}}{2016.07 - 2017.07}
    \begin{onehalfspacing}
        \datedline{\textbf{WhatsMODE网站全栈开发}}{2016.12 - 2017.07}
        \begin{itemize}
            \item 基于开源项目Magento2进行二次定制开发
            \item 前端、后台、生产部署全栈开发
            \item 集成网红管理系统,为网红定制网红店铺
        \end{itemize}
    \end{onehalfspacing}
    \blankline{ }

    \begin{onehalfspacing}
        \datedline{\textbf{MODE APP后台开发}}{2016.07 - 2016.11}
        \begin{itemize}
            \item 用户模块,集成RBAC权限管理系统
            \item 独立完成搜索模块,ES Client API集成开发
            \item 社交模块,用户评论、点赞以及互动
        \end{itemize}
    \end{onehalfspacing}
    \blankline{ }

    \datedsubsection{\textbf{搜狐北京研发中心} \hfill \textit{大数据研发工程师(实习)}}{2016.03 - 2016.05}
    \begin{onehalfspacing}
        \datedline{\textbf{大数据平台日志分析}}{\textit{Scala + Spark + Kafka}}
        \begin{itemize}
            \item 利用Kafka每天定时收集40-50G规模日志数据,拥有TB级日志数据处理能力
            \item 支持离线(Spark)、实时(Spark Streaming)两种处理模式
            \item 采用迭代式开发,日志异常报错实时邮件通知
        \end{itemize}
    \end{onehalfspacing}
    \blankline{ }

    \datedsubsection{\textbf{58同城(北京总部)} \hfill \textit{Java 开发工程师(实习)}}{2015.07 - 2015.09}
    \begin{onehalfspacing}
        \datedline{\textbf{会员增值业务产品开发}}{\textit{MVC + Memcached}}
        \begin{itemize}
            \item 掌握MVC框架产品开发流程和产品上线维护流程
            \item 大型企业数据库分库分表设计,Memcached分布式缓存使用
            \item 根据业务逻辑对外提供接口和微服务
        \end{itemize}
    \end{onehalfspacing}
    \blankline{ }

    \section{\faBriefcase\ 个人项目}
    \datedsubsection{\textbf{Bing壁纸爬虫} \hfill \textit{\href{https://github.com/benheart/BingGallery}{https://github.com/benheart/BingGallery}}}{2018.01}
    \begin{onehalfspacing}
        \begin{itemize}
            \item Python爬虫学习
            \item 分析HTTP数据报文,定位关键请求
            \item 编写全量爬取Bing壁纸爬虫
            \item 文档整理总结,并在博客、社交网站分享
        \end{itemize}
    \end{onehalfspacing}

    \datedsubsection{\textbf{Magento2中文化} \hfill \textit{\href{https://github.com/benheart/magento2\_zh\_hans\_cn}{https://github.com/benheart/magento2\_zh\_hans\_cn}}}{2017.04 - 2017.06}
    \begin{onehalfspacing}
        \begin{itemize}
            \item 学习并掌握Magento2国际化(i18n)原理
            \item 基于\href{https://crowdin.com/project/magento-2/zh-CN}{官方翻译项目}制作Magento2中文语言包,
            \item 利用Compose进行版本管理,实时同步分发语言包
        \end{itemize}
    \end{onehalfspacing}

    \datedsubsection{\textbf{微博爬虫} \hfill \textit{\href{https://github.com/benheart/microblog-crawler}{https://github.com/benheart/microblog-crawler}}}{2016.03 - 2016.06}
    \begin{onehalfspacing}
        \begin{itemize}
            \item 爬取500多万微博用户资料、社交关系
            \item 数据库分库分表,存储用户资料和队列信息
            \item 多线程爬取(互斥锁),支持多Cookie和断点爬取
            \item 爬取热门话题和特定用户群体,利用K-近邻算法处理Json格式数据进行用户分类
            \item 利用Matplotlib绘图库和Gephi可视化工具进行数据展示,构建人物关系图谱
        \end{itemize}
    \end{onehalfspacing}
    \blankline{ }

    \section{\faCogs\ 专业技能}
    \begin{itemize}[parsep=0.5ex]
        \item 英语: CET-6
        \item 计算机职业资格认证: 210分,排名17.75\% / 共3819人
        \item 编程语言: Java, Python, Mysql
        \item 研究方向: 后台开发、网络爬虫、数据挖掘与分析
    \end{itemize}
    \blankline{ }

    \section{\faTrophy\ 荣誉奖励}
    \datedline{\textit{哈尔滨工业大学“优秀学生干部”}}{2014.12}
    \datedline{\textit{大一年度科技创新项目获得全院二等奖}}{2013.06}
    \blankline{ }

    \section{\faLink\ 网站链接}
    % increase linespacing [parsep=0.5ex]
    \begin{itemize}[parsep=0.5ex]
        \item GitHub网址: https://github.com/benheart
        \item 技术博客网址: https://www.geeeeeeek.com/
    \end{itemize}

\end{document}
